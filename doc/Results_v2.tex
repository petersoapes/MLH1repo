\documentclass[]{article}
\usepackage{lmodern}
\usepackage{amssymb,amsmath}
\usepackage{ifxetex,ifluatex}
\usepackage{fixltx2e} % provides \textsubscript
\ifnum 0\ifxetex 1\fi\ifluatex 1\fi=0 % if pdftex
  \usepackage[T1]{fontenc}
  \usepackage[utf8]{inputenc}
\else % if luatex or xelatex
  \ifxetex
    \usepackage{mathspec}
  \else
    \usepackage{fontspec}
  \fi
  \defaultfontfeatures{Ligatures=TeX,Scale=MatchLowercase}
\fi
% use upquote if available, for straight quotes in verbatim environments
\IfFileExists{upquote.sty}{\usepackage{upquote}}{}
% use microtype if available
\IfFileExists{microtype.sty}{%
\usepackage{microtype}
\UseMicrotypeSet[protrusion]{basicmath} % disable protrusion for tt fonts
}{}
\usepackage[margin=1in]{geometry}
\usepackage{hyperref}
\hypersetup{unicode=true,
            pdftitle={MLH1 Results Outline},
            pdfborder={0 0 0},
            breaklinks=true}
\urlstyle{same}  % don't use monospace font for urls
\usepackage{graphicx,grffile}
\makeatletter
\def\maxwidth{\ifdim\Gin@nat@width>\linewidth\linewidth\else\Gin@nat@width\fi}
\def\maxheight{\ifdim\Gin@nat@height>\textheight\textheight\else\Gin@nat@height\fi}
\makeatother
% Scale images if necessary, so that they will not overflow the page
% margins by default, and it is still possible to overwrite the defaults
% using explicit options in \includegraphics[width, height, ...]{}
\setkeys{Gin}{width=\maxwidth,height=\maxheight,keepaspectratio}
\IfFileExists{parskip.sty}{%
\usepackage{parskip}
}{% else
\setlength{\parindent}{0pt}
\setlength{\parskip}{6pt plus 2pt minus 1pt}
}
\setlength{\emergencystretch}{3em}  % prevent overfull lines
\providecommand{\tightlist}{%
  \setlength{\itemsep}{0pt}\setlength{\parskip}{0pt}}
\setcounter{secnumdepth}{0}
% Redefines (sub)paragraphs to behave more like sections
\ifx\paragraph\undefined\else
\let\oldparagraph\paragraph
\renewcommand{\paragraph}[1]{\oldparagraph{#1}\mbox{}}
\fi
\ifx\subparagraph\undefined\else
\let\oldsubparagraph\subparagraph
\renewcommand{\subparagraph}[1]{\oldsubparagraph{#1}\mbox{}}
\fi

%%% Use protect on footnotes to avoid problems with footnotes in titles
\let\rmarkdownfootnote\footnote%
\def\footnote{\protect\rmarkdownfootnote}

%%% Change title format to be more compact
\usepackage{titling}

% Create subtitle command for use in maketitle
\newcommand{\subtitle}[1]{
  \posttitle{
    \begin{center}\large#1\end{center}
    }
}

\setlength{\droptitle}{-2em}

  \title{MLH1 Results Outline}
    \pretitle{\vspace{\droptitle}\centering\huge}
  \posttitle{\par}
    \author{}
    \preauthor{}\postauthor{}
      \predate{\centering\large\emph}
  \postdate{\par}
    \date{null}

\usepackage{booktabs}
\usepackage{longtable}
\usepackage{array}
\usepackage{multirow}
\usepackage{wrapfig}
\usepackage{float}
\usepackage{colortbl}
\usepackage{pdflscape}
\usepackage{tabu}
\usepackage{threeparttable}
\usepackage{threeparttablex}
\usepackage[normalem]{ulem}
\usepackage{makecell}
\usepackage{xcolor}

\begin{document}
\maketitle

\includegraphics{Results_v2_files/figure-latex/three.plots-1.pdf}

\begin{tabular}{l|l|l|r|r|r|r|r|r|r}
\hline
subsp & strain & sex & Nmice & Ncells & mean\_co & cV & var & sd & se\\
\hline
Dom & WSB & female & 14 & 184 & 25 & 15 & 13.1 & 3.6 & 0.27\\
\hline
Dom & WSB & male & 11 & 222 & 23 & 11 & 7.2 & 2.7 & 0.18\\
\hline
Dom & G & female & 12 & 318 & 28 & 15 & 17.5 & 4.2 & 0.24\\
\hline
Dom & G & male & 18 & 355 & 23 & 11 & 6.9 & 2.6 & 0.14\\
\hline
Dom & LEW & female & 9 & 147 & 27 & 18 & 23.3 & 4.8 & 0.40\\
\hline
Dom & LEW & male & 10 & 253 & 24 & 13 & 9.6 & 3.1 & 0.20\\
\hline
Cast & CAST & female & 1 & 1 & 26 & NA & NA & NaN & NaN\\
\hline
Cast & CAST & male & 2 & 44 & 22 & 10 & 5.2 & 2.3 & 0.34\\
\hline
Musc & PWD & female & 15 & 222 & 26 & 14 & 14.0 & 3.7 & 0.25\\
\hline
Musc & PWD & male & 8 & 161 & 29 & 11 & 9.8 & 3.1 & 0.25\\
\hline
Musc & MSM & female & 14 & 300 & 28 & 16 & 19.4 & 4.4 & 0.25\\
\hline
Musc & MSM & male & 7 & 166 & 30 & 10 & 9.7 & 3.1 & 0.24\\
\hline
Musc & MOLF & female & 1 & 21 & 28 & 15 & 17.9 & 4.2 & 0.92\\
\hline
Musc & MOLF & male & 6 & 119 & 23 & 11 & 6.4 & 2.5 & 0.23\\
\hline
Musc & SKIVE & female & 1 & 32 & 26 & 12 & 9.8 & 3.1 & 0.55\\
\hline
Musc & SKIVE & male & 3 & 86 & 26 & 10 & 7.4 & 2.7 & 0.29\\
\hline
Musc & KAZ & female & 9 & 184 & 26 & 16 & 16.0 & 4.0 & 0.30\\
\hline
Musc & KAZ & male & 13 & 264 & 23 & 13 & 9.2 & 3.0 & 0.19\\
\hline
Spretus & SPRET & female & 2 & 2 & 26 & 11 & 8.0 & 2.8 & 2.00\\
\hline
Spretus & SPRET & male & 5 & 103 & 24 & 10 & 6.2 & 2.5 & 0.25\\
\hline
Spic & SPIC & female & 6 & 97 & 28 & 16 & 19.5 & 4.4 & 0.45\\
\hline
Spic & SPIC & male & 4 & 133 & 26 & 11 & 7.7 & 2.8 & 0.24\\
\hline
\end{tabular}

\section{genome-wide recombination rate estimates for both
sexes}\label{genome-wide-recombination-rate-estimates-for-both-sexes}

We used counts of MLH1 foci per cell to estimate genome-wide
recombination rates in 14 wild-derived inbred strains sampled from three
subspecies of house mice: \emph{M. m. domesticus}, \emph{M. m. musculus}
and \emph{M. m. molossinus}. Mean MLH1 foci counts for 166 mice were
quantified from an average of 19.73 spermatocytes per male (for a total
of 1867 spermatocytes) and 18.54 oocytes per female (for a total of 1409
oocytes).

Graphical comparisons between the two sexes reveal several patterns
(Figure 1A). First, recombination rate differs between females and males
in most strains. Second, the direction and magnitude of heterochiasmy
varies among strains. Although the majority of strains show higher
recombination rates in females (following the pattern in laboratory
mice), two musculus strains and one molossinus strain exhibit
male-biased heterochiasmy. Hence, relative recombination rates in the
two sexes are evolutionarily labile.

The general patterns of heterochiasmy in house mouse are displayed in
Figure 1A. Taking note of the direction and the magnitude of the sex
differences, our results confirm two general patterns: i) genome-wide
recombination rates averages are greater in females compared to males
(female biased heterochiasmy) and ii) the degree of heterochiasmy
(Female:Male ratio) is generally low, ranging from 1.22 in
\emph{domesticus\textsuperscript{G}} to 1.06 in
\emph{domesticus\textsuperscript{LEW}}. Three notable exceptions of male
biased heterochiasmy are the strains
\emph{musculus\textsuperscript{PWD}},
\emph{molossinus\textsuperscript{MSM}} and
\emph{musculus\textsuperscript{SKIVE}}, with heterochiasmy values of
0.91, 0.93 and 0.99 respectively.

Separately examining the mouse means of MLH1 foci per cell for each sex
points to distinct patterns of variation (Figure 1B-C). Female
recombination rates are evenly distributed around the sex-wide mean of
approximately 25 MLH1 foci per cell (Figure 1B). In stark contrast,
males separate more clearly into two groups of strains with high rates
(near 30 MLH1 foci per cell) and low rates (near 23 MLH1 foci per cell)
(Figure 1C).

\section{Partitioning variation in recombination
rate}\label{partitioning-variation-in-recombination-rate}

To situate variation in recombination rate within an evolutionary
framework, we fit a series of models including subspecies, strain, and
sex, to mean MLH1 foci counts from 187 mice. We began with a full mixed
model (M1, see Methods), which showed that strain (random effect p
\textless{} 0 -- 10\^{}\{-6\} someSmallNumber), sex (p =
1.55\times 10\^{}\{-8\}), subspecies (p=1.72\times 10\^{}\{-4\}), and
subspecies sex (p = 3.1\times 10\^{}\{-5\}) each significantly affect
recombination rate.

After a general linear model including all factors as fixed effects (M2)
revealed only weak contributions of subspecies, we focused on additional
models designed to illuminate the role of strain and sex. A general
linear model with these two variables (M3) identified two strains with
particularly strong effects on recombination rate:
musculus\textsuperscript{MSM} (p= 3.99\times 10\^{}\{-6\}) and
domesticus\textsuperscript{G} (1.04\times 10\^{}\{-6\}). In addition,
two strains exhibit strain-by-sex interactions:
molossinus\textsuperscript{MSM} (1.26\times 10\^{}\{-4\}) and
musculus\textsuperscript{PWD} (3.86\times 10\^{}\{-4\}).

We next fit general linear models separately for 192 males and 144
females (M4; see methods). In the male dataset, three strains
significantly affect recombination rate: musculus\textsuperscript{PWD}
((glm; p = 6.31\times 10\^{}\{-8\}; effect = 6.11 foci), and
musculus\textsuperscript{SKIVE} (glm; p = 0.01; 0; effect = 3.8), and
\emph{molossinus\textsuperscript{MSM}} (glm; p=2.42\times 10\^{}\{-12\};
effect 6.99).

These three strains point to rapid evolution in recombination rate in
spermatocytes; we subsequently refer to them collectively as
``high-recombination'' strains. Analysis of the female dataset points to
four strains with significant effects on recombination rate:
domesticus\textsuperscript{G} (p = 2.5\times 10\^{}\{-6\}),
molossinus\textsuperscript{MSM} (p = 6.24\times 10\^{}\{-6\}),
domesticus\textsuperscript{LEW} (p = 0.01), and
musculus\textsuperscript{PWD} (p= 0.02). Strain effect sizes in females
are modest in magnitude (ranging from 1 to 4 foci) compared to those in
males. Together, these results demonstrate heritable differences in the
genome-wide recombination rate evolving in a highly sex-specific manner
over short evolutionary timescales.

\section{Within mouse variance in CO count per
cell}\label{within-mouse-variance-in-co-count-per-cell}

Counting MLH1 foci in multiple oocytes for each female and multiple
spermatocytes for each male allowed us to examine determinants of the
within-mouse variance in recombination rate. To do this, we considered
the same models as above, but replaced mean MLH1 foci count with
within-mouse variance in MLH1 foci count as the dependent variable. Sex
is the only variable that significantly affects recombination rate in
both the mixed model (M1) (p \textless{} 10\^{}\{-6\} -- 0) and general
linear models (M2) (p = 0.03) and M3 (p = 0.03).

In general, females have almost twice as much variance in MLH1 foci per
cell compared to males (Figure 1). Since estimates of within-mouse
variance may be more susceptible to technical error from the staining
protocol, we repeated the analyses using a subset of cells with higher
quality scores (quality score 1 and 2, see Materials and Methods). The
results are similar: sex is the strongest effect (M1 p \textless{}
10\^{}\{-6\} 0; M2 p = 2.3\times 10\^{}\{-4\}; also M3 p =
2.28\times 10\^{}\{-4\}). When both quality-curated and full datasets
are considered, strain does not significantly and consistently affect
variance in MLH1 foci count per cell in either sex. These results
suggest that within-mouse variance in recombination rate evolves
independently of mean recombination rate.

 \textbf{re-read Lenzi et al -- heterogeneity in oogenesis}

\section{DSB. Evolution of genome-wide recombination rate is associated
with evolution of double strand
breaks}\label{dsb.-evolution-of-genome-wide-recombination-rate-is-associated-with-evolution-of-double-strand-breaks}

\includegraphics{Results_v2_files/figure-latex/MAIN.DMC1.plot-1.pdf}

In an attempt to localize the male-specific evolution of crossover
number to steps of the meiotic pathway, we counted foci from a marker
for double strand breaks (DSBs), DMC1, in prophase spermatocytes. DMC1
foci were scored from a total of 0 early zygotene-stage and 0 late
zygotene-stage spermatocytes from three low-recombination strains
(musculus\textsuperscript{KAZ} , domesticus\textsuperscript{WSB} , and
domesticus\textsuperscript{G}) and two high-recombination strains
(musculus\textsuperscript{PWD} and molossinus\textsuperscript{MSM}).

The high-recombination strains have significantly more DMC1 foci than
the low-recombination strains in early zygotene cells (t-test,
p\textless{} -- 10\^{}\{-6\} -- 0). In contrast, the two strain groups
do not differ in DMC1 foci counted in late zygotene cells (t-test, p =
0.66 --- 0.66).

After DSB formation, DSBs are repaired as either non-crossovers (NCO) or
as crossovers (COs), with the vast majority being repaired as NCOs. Thus
the ratio of CO:DSB is a partial indicator of the proportion of DSBs
which are designated as COs. The ratios, calculated for DMC1 means from
both stages, are not significantly different between the high and low
strain groups (t-test, p = 0.94 and p = 0.11 for early zygotene and late
zygotene ratios, respectively).

This comparison raises the possibility that the evolution of crossover
number is primarily due to processes that precede the
crossover/non-crossover decision. (Cole et al. 2012) suggest that early
zygotene counts for DMC1 are most relevant for predicting the final CO
number due to the crossover homeostasis process. This result, combined
with the big difference in MLH1 count between high and low strains,
might indicate that the CO/NCO decision contributes to strain
differences in recombination rate.

\section{Evolution of genome-wide recombination rate is reflected at the
single chromosome
level}\label{evolution-of-genome-wide-recombination-rate-is-reflected-at-the-single-chromosome-level}

\includegraphics{Results_v2_files/figure-latex/Chrm.props.plot2-1.pdf}

To examine the connection between evolution of the genome-wide
recombination rate and changes to the recombination landscape, we used
an image analysis pipeline to measure properties of single bivalents
((Peterson, Miller, and Payseur 2019)).

 This algorithm substantially speeds the accurate measurement of
bivalents, but has the limitation that not all bivalents per cell can be
isolated due to overlapping bivalents. In this dataset, isolation rates
per cell range from 0.51 (molossinus\textsuperscript{MSM} male) to 0.72
(musculus\textsuperscript{KAZ} female).

From the total set of cell images, 10458 bivalent objects were isolated
by the image analysis software. After a human curation step ((Peterson,
Miller, and Payseur 2019)), 9829 single-bivalent observations remained.
We assume that the isolation of bivalents within cells is unbiased.

Given the large number of single-bivalent observations, we assume that
each of the datasets are equally representative of general patterns.

An additional challenge of the MLH1 framework is that the identities of
individual autosomes and the XX in females cannot be easily obtained
(the male XY is distinct).

Ninety-six percent of single bivalents in our pooled dataset (n = 34982)
have either one or two crossovers (Figure X). The proportion of
one-crossover (1CO) to two-crossover (2CO) bivalents distinguishes the
high vs.~low recombining strains (Figure X). High-recombination strains
are enriched for 2CO bivalents at the expense of 1CO bivalents:
proportions of 2CO bivalents are 0.33 -- 0.33 (
\emph{musculus\textsuperscript{SKIVE}} ) in
musculus\textsuperscript{SKIVE}, 0.44 -- 0.44 in
musculus\textsuperscript{PWD}, and 0.53 in
molossinus\textsuperscript{MSM}.

Following patterns in the genome-wide recombination rate, male
musculus\textsuperscript{PWD} and male molossinus\textsuperscript{MSM}
have 2CO proportions that are more similar to each other than to strains
from their own subspecies (chi-square tests;
musculus\textsuperscript{PWD} vs.~musculus\textsuperscript{KAZ} p =
3.15\times 10\^{}\{-33\}; molossinus\textsuperscript{MSM}
vs.~molossinus\textsuperscript{MOLF} p = 4.72\times 10\^{}\{-13\}).

Next, we use this single-bivalent dataset to focus on aspects of the
recombination landscape along chromosomes. We address two main
questions. First, which traits are sexually dimorphic? Second, which
traits differ between males from high-recombination
vs.~low-recombination strains?

\section{Q1. Sex Differences in the Length of the Synaptonemal
Complex}\label{q1.-sex-differences-in-the-length-of-the-synaptonemal-complex}

\includegraphics{Results_v2_files/figure-latex/Q1.SC.show.short.biv-1.pdf}

\includegraphics{Results_v2_files/figure-latex/Q1.tot.SC_show-1.pdf}

In many mammalian species, the synaptonemal complex (SC) is longer in
females. Yet, the majority of these observations come from species with
female-biased heterochiasmy. Our wider survey of recombination in house
mice provides an opportunity to determine whether sex differences in
chromatin compaction (SC length) are reversed when heterochiasmy is
male-biased. In addition, if SC length is a strong determinant of the
genome-wide recombination rate, male musculus\textsuperscript{PWD} and
male molossinus\textsuperscript{MSM} should have longer SC lengths than
the other strains.

Short bivalents are significantly longer in females than males in all
strains (t-test; p \textless{} 0.05) except
musculus\textsuperscript{SKIVE}. The female:male ratio ranges across
strains from 1.15 (musculus\textsuperscript{MSM}) to 1.49
(domesticus\textsuperscript{WSB}).

Females have significantly longer total SC than males in each strain
tested separately (t-test; p \textless{} 0.05). That females have longer
SCs is also supported by mixed models and general linear models with
covariates, which identify sex as the most consistently significant
effect (p \textless{} 0.05). Additionally, there are some significant
subspecies and strain effects (p \textless{} 0.05), indicating that SC
length has evolved among strains and subspecies.

In summary, two approaches for measuring and analyzing SC length
indicate that females have longer SCs (chromosome axes), even in strains
where males have more MLH1 foci per cell. Furthermore, joint
consideration of MLH1 foci and total SC length suggests that males from
high-recombination strains have less ``space'' in which to place their
additional crossovers.

\section{Q1. Sex Differences in the Positions of Single
Crossovers}\label{q1.-sex-differences-in-the-positions-of-single-crossovers}

In most of the strains we surveyed, the majority of bivalents are
observed to contain one crossover (focus). In all strains, the landscape
across 1CO bivalents is significantly different in females and males.
Normalized foci positions tend to be more central in females 0.56 and
closer to the telomere in males 0.68 (t-test; p =
2.92\times 10\^{}\{-22\}). Sex is also the most significant effect on
focus position in a mixed model (M1: p = 1.26\times 10\^{}\{-25\}) and
in general linear models (M2: p = 1.33\times 10\^{}\{-7\}; M3: p =
1.33\times 10\^{}\{-7\}). These sex differences in the placement of foci
on 1CO bivalents follow a pattern observed across a variety of mammalian
species (Sardell and Kirkpatrick 2020).

\section{Q1. Sex Differences in CO Interference
(IFD)}\label{q1.-sex-differences-in-co-interference-ifd}

\begin{figure}
\centering
\includegraphics{Results_v2_files/figure-latex/Q1.IFD_load.triangles-1.pdf}
\caption{Example of Sex differences in IFD distributions}
\end{figure}

There is no strong signal of sex differences in raw mean inter-focal
distances (IFD\textsuperscript{raw}) across the full set of strains. A
marginally significant difference between the sexes (t-test; p = 0.07 --
0.07) is driven by one strain (t-test without
domesticus\textsuperscript{G}; p = 0.27 -- 0.02). This result indicates
that females and males exhibit a similar level of interference when it
is measured in physical (SC) units. In contrast, males have
significantly longer normalized mean inter-focal distances
(IFD\textsuperscript{norm}) than females in seven out of eight strains
(t-tests; p \textless{} 0.02 -- 1.49\times 10\^{}\{-12\}), with only
musculus\textsuperscript{KAZ} showing no difference (t-test; p = 0.33 --
0.33). Examination of IFD\textsuperscript{norm} distributions indicates
that female IFD\textsuperscript{norm} values are centered at
approximately 50\% and show a slight enrichment of low (\textless{}25\%)
values, whereas males are enriched for higher values. Mixed models and
general linear models of IFD\textsuperscript{norm} support the inference
of stronger interference in males: sex is the most significant variable
(M1 - LRT \textbf{random}: p = 6.7410\^{}\{-14\} --
6.74\times 10\^{}\{-14\} glm \emph{M2} 0.01 \emph{M3} 0.01). When
interference is measured in physical SC units
(IFD\textsuperscript{raw}), the differences between sexes is low and
only slightly significant (data not shown). In summary, controlling for
differences in chromatin compaction (SC length) using IFDnorm indicates
that interference is stronger in males, whereas consideration of
IFD\textsuperscript{raw} shows that the sexes exhibit a similar level of
interference on the physical (SC) scale. While the comparisons of the
IFD\textsuperscript{norm} metrics can reveal more general recombination
landscape patterns while controlling for the underlying differences in
chromatin compaction and SC length.

\section{Additional Determinants of Genome-wide Recombination Rate
Evolution in
Males}\label{additional-determinants-of-genome-wide-recombination-rate-evolution-in-males}

The next section is meant to focusNext, we used the contrast between
high-recombination strains (musculus\textsuperscript{PWD} ,
musculus\textsuperscript{SKIVE}, and molossinus\textsuperscript{MSM})
and low-recombination strains to identify features of the recombination
landscape associated with evolutionary transitions in the genome-wide
recombination rate. on the greater aspect of variation in mean MLH1
counts per cell the high recombining strains (
musculus\textsuperscript{PWD} , musculus\textsuperscript{SKIVE}, and
molossinus\textsuperscript{MSM}) to the low recombing strains. The main
objective of this section is to test for significant correlations
between features of the recombination landscape and the evolution of
mean MLH1 foci per cell. In comparing the male specific single bivalent
based metrics the first creteria for analysis is a significant
differences between the high and low recombining groups and the second
step is testing for significant subspecies and strain effects in the
same models which used mean MLH1 count per cell.

Previous empirical work suggests basic predictions for the relationship
between the genome wide recombination rate and two aspects of the
recombination landscape. \textbf{SC length is expected to be positively
associated with genome-wide recombination rate because of the loop-axis
strucutre} (zickler Kleckner 1999, Merier 2015?). Crossover interference
strength is expected to be negatively \textbf{associated with
genome-wide recombination rate since interference is defined as the
non-random spacing between crossovers on the same chromosome}.

Following this logic we predict (1) musculus\textsuperscript{PWD} will
have greater SC length and weaker interference than
musculus\textsuperscript{SKIVE}, which in turn will have longer SC and
weaker interference compared to the other \emph{musculus} strains, (2)
molossinus\textsuperscript{MSM} will have longer SC and weaker
interference compared to molossinus\textsuperscript{MOLF}, and 3)
\emph{domesticus} strains will have similar SC length and crossover
interference.

\section{Q2 SC Length}\label{q2-sc-length}

\includegraphics{Results_v2_files/figure-latex/Q2_tot.sc_plot-1.pdf}

\includegraphics{Results_v2_files/figure-latex/Q2.SC_MLH1.by.totalSC-1.pdf}

Confirming the basic predictions, there is a positive correlation
between mean MLH1 foci per cell and total SC ( Spearmans' r = 0.48;

p = 2.24\times 10\^{}\{-10\}). Nevertheless, mean total SC only
partially differentiates high-recombination and low-recombination
strains (Figure X). \textbf{Bret's note on positive correlation, 'A
correlation across mice, across strains, or both?} While
high-recombination strains have significantly more SC area in the total
dataset (t-test; p = 0.01 -- 0.01), separate tests by strain show that
only within \emph{molossinus} are high- and low-recombination strains
significantly different (t-test; molossinus: p = 0.03 -- 0.03; musculus:
p = 0.87 -- 0.87). \textbf{the values above are diferent between
editions}

Additionally, the means for the reduced (short and long) bivalent
datasets are not significantly different between high-recombination and
low-recombination strains (t-test; short: p = 0.88 -- 0.88; long: p =
0.18 -- 0.18).

\textbf{the values between versions are slightly different} In a general
linear model with total SC as the dependent variable, two subspecies
effects are significant (p = musculus 1.2410\^{}\{-6\} --
1.24\times 10\^{}\{-6\}, molossinus p = 10\^{}\{-6\} -- 10\^{}\{-6\}).
In general linear models with reduced bivalent means as dependent
variables, several subspecies and strain effects reach significance (p
\textless{} 0.05) but they are not consistent across models, indicating
to some extent the chromatin compaction evolution is decoupled from
evolution in mean MLH1 foci per cell. \textbf{Bret's comment; How can
you differentiate between this biological conclusion and uncertainty due
to the size of the datasets and statistical approaches?}

\section{Q2.1CO rec landscape evolution is decoupled from gwRR
evolution}\label{q2.1co-rec-landscape-evolution-is-decoupled-from-gwrr-evolution}

\includegraphics{Results_v2_files/figure-latex/Q2.1CO.show_plots-1.pdf}

The normalized 1CO position is not significantly different between
high-recombination and low-recombination strains for the total pooled
data (t-test; p = 0.24 -- 0.24) and also when examined within subspecies
(t-test; p = 0.41 -- 0.41 and p = 0.07 -- 0.07 for \emph{musculus} and
\emph{molossinus}, respectively). \textbf{While there are significant
strain effects for domesticus\textsuperscript{WSB} and
molossinus\textsuperscript{MOLF} in a general linear model with
normalized position as the dependent variable (Figure X)}, \emph{this
evolution of the 1CO positioning is decoupled from the total genome-wide
recombination rate.}

\section{Q2 Evolution of interference is associated with genome wide
recombination rate
evolution}\label{q2-evolution-of-interference-is-associated-with-genome-wide-recombination-rate-evolution}

Mouse averages for both IFD\textsuperscript{raw} and
IFD\textsuperscript{norm} are significantly longer in high-recombination
strains (t-test; IFD\textsuperscript{norm}: p = 7.7410\^{}\{-7\} --
7.74\times 10\^{}\{-7\}; IFD\textsuperscript{raw}: p = 8.7810\^{}\{-6\}
-- 8.78\times 10\^{}\{-6\}). This pattern is confirmed by separate
comparisons within \emph{musculus} (t-test; IFD\textsuperscript{norm}: p
=2.0410\^{}\{-5\} -- 2.04\times 10\^{}\{-5\}; IFD\textsuperscript{raw}:
p = 1.9410\^{}\{-4\} -- 1.94\times 10\^{}\{-4\}) and within
\emph{molossinus} (IFD\textsuperscript{norm}: p= 0.17 -- 0.17;
IFD\textsuperscript{raw}: p = 0.08 -- 0.08). Similar results are seen
with general linear models for both IFD\textsuperscript{raw} and
IFD\textsuperscript{norm}: only effects associated with
high-recombination strains are significant (p \textless{} 0.05).

That IFD\textsuperscript{raw} and IFD\textsuperscript{norm} show similar
patterns eliminates variation in SC lengths and bivalent sizes as
primary explanations. We determined that the main difference in
IFD\textsuperscript{norm} distributions between high-recombination and
low-recombination strains is an enrichment of IFD\textsuperscript{norm}
values under 30\% --

in low-recombination strains. The frequency of IFD\textsuperscript{norm}
values that fall below 30\% ranges from 8.2\%
(domesticus\textsuperscript{G}) to 16\% (musculus\textsuperscript{KAZ})
in low-recombination strains, whereas high-recombination strains all
show such frequencies below 5\% (0\%, 1.3\%, and 3.3\% for
musculus\textsuperscript{SKIVE}, molossinus\textsuperscript{MSM}, and
musculus\textsuperscript{PWD}, respectively).

In summary, the level of interference is a significant predictor of
evolution in the genome-wide recombination rate, but SC length and
crossover position on 1CO bivalents are not. However, the pattern is in
the opposite direction to our prediction: high-recombination strains
have stronger interference. At least on 2CO bivalents, crossovers are
spaced further apart when the genome-wide recombination rate is higher.

\section{Q2 Summary}\label{q2-summary}

\textbf{move main themes to conclusion}

Our results show that the greater crossover interference is the
strongest single bivalent-based predictor for the observed rapid
evolution of mean MLH1 foci per cell. While these results do not conform
initial predictions for how a recombination landscapes would accommodate
more crossovers, the increased strength of interference aligns with our
results on sex differences. The typical recombination landscapes for
males and females results in divergence in the proportion of linked
sites along chromosomes which segregate together. The stronger
interference of the 2CO bivalents in the high recombining strains
accentuates this effect. The measures of DSB and some comparisons of SC
length between high and low recombining strains suggest that the SC
length have evolved to be longer in high recombining strains, however
this evolution of SC length is partially decoupled from the number of
crossovers since similar amounts of SC length evolution are seen in low
recombining strains.

\textless{}Bret: Are you sure this summary is necessary? Maybe cover
this material in the Discussion instead?\textgreater{}

\section*{References}\label{references}
\addcontentsline{toc}{section}{References}

\hypertarget{refs}{}
\hypertarget{ref-cole2012}{}
Cole, Francesca, Liisa Kauppi, Julian Lange, Ignasi Roig, Raymond Wang,
Scott Keeney, and Maria Jasin. 2012. ``Homeostatic Control of
Recombination Is Implemented Progressively in Mouse Meiosis.''
\emph{Nature Cell Biology} 14 (4). Nature Publishing Group: 424--30.

\hypertarget{ref-peterson2019}{}
Peterson, April L, Nathan D Miller, and Bret A Payseur. 2019.
``Conservation of the Genome-Wide Recombination Rate in White-Footed
Mice.'' \emph{Heredity} 123 (4). Nature Publishing Group: 442--57.

\hypertarget{ref-sardell_sex_2020}{}
Sardell, Jason M., and Mark Kirkpatrick. 2020. ``Sex Differences in the
Recombination Landscape.'' \emph{The American Naturalist} 195 (2):
361--79. doi:\href{https://doi.org/10.1086/704943}{10.1086/704943}.


\end{document}
