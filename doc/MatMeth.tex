\documentclass[]{article}
\usepackage{lmodern}
\usepackage{amssymb,amsmath}
\usepackage{ifxetex,ifluatex}
\usepackage{fixltx2e} % provides \textsubscript
\ifnum 0\ifxetex 1\fi\ifluatex 1\fi=0 % if pdftex
  \usepackage[T1]{fontenc}
  \usepackage[utf8]{inputenc}
\else % if luatex or xelatex
  \ifxetex
    \usepackage{mathspec}
  \else
    \usepackage{fontspec}
  \fi
  \defaultfontfeatures{Ligatures=TeX,Scale=MatchLowercase}
\fi
% use upquote if available, for straight quotes in verbatim environments
\IfFileExists{upquote.sty}{\usepackage{upquote}}{}
% use microtype if available
\IfFileExists{microtype.sty}{%
\usepackage{microtype}
\UseMicrotypeSet[protrusion]{basicmath} % disable protrusion for tt fonts
}{}
\usepackage[margin=1in]{geometry}
\usepackage{hyperref}
\hypersetup{unicode=true,
            pdftitle={Material Methods},
            pdfborder={0 0 0},
            breaklinks=true}
\urlstyle{same}  % don't use monospace font for urls
\usepackage{graphicx,grffile}
\makeatletter
\def\maxwidth{\ifdim\Gin@nat@width>\linewidth\linewidth\else\Gin@nat@width\fi}
\def\maxheight{\ifdim\Gin@nat@height>\textheight\textheight\else\Gin@nat@height\fi}
\makeatother
% Scale images if necessary, so that they will not overflow the page
% margins by default, and it is still possible to overwrite the defaults
% using explicit options in \includegraphics[width, height, ...]{}
\setkeys{Gin}{width=\maxwidth,height=\maxheight,keepaspectratio}
\IfFileExists{parskip.sty}{%
\usepackage{parskip}
}{% else
\setlength{\parindent}{0pt}
\setlength{\parskip}{6pt plus 2pt minus 1pt}
}
\setlength{\emergencystretch}{3em}  % prevent overfull lines
\providecommand{\tightlist}{%
  \setlength{\itemsep}{0pt}\setlength{\parskip}{0pt}}
\setcounter{secnumdepth}{0}
% Redefines (sub)paragraphs to behave more like sections
\ifx\paragraph\undefined\else
\let\oldparagraph\paragraph
\renewcommand{\paragraph}[1]{\oldparagraph{#1}\mbox{}}
\fi
\ifx\subparagraph\undefined\else
\let\oldsubparagraph\subparagraph
\renewcommand{\subparagraph}[1]{\oldsubparagraph{#1}\mbox{}}
\fi

%%% Use protect on footnotes to avoid problems with footnotes in titles
\let\rmarkdownfootnote\footnote%
\def\footnote{\protect\rmarkdownfootnote}

%%% Change title format to be more compact
\usepackage{titling}

% Create subtitle command for use in maketitle
\providecommand{\subtitle}[1]{
  \posttitle{
    \begin{center}\large#1\end{center}
    }
}

\setlength{\droptitle}{-2em}

  \title{Material Methods}
    \pretitle{\vspace{\droptitle}\centering\huge}
  \posttitle{\par}
    \author{}
    \preauthor{}\postauthor{}
      \predate{\centering\large\emph}
  \postdate{\par}
    \date{2019-09-18}


\begin{document}
\maketitle

{
\setcounter{tocdepth}{5}
\tableofcontents
}
\section{Mouse Husbandry}\label{mouse-husbandry}

To access natural genetic variation for \emph{Mus musculus}, wild
derived inbred strains were used. The Mus musculus PWD/PhJ, PERC/Eij,
WSB/EiJ, LEWE/EiJ, MSM/MsJ, MOLF/EiJ, CAST/EiJ, CZECHII/EiJ were
purchased from Jackson labs (Maine, USA). The strains of KAZ/TUA,
TOM/TUA, AST/TUA, and HMI/TUA were cryo-derived from Biological Resource
Center (BRC) at Riken (Ibaraki, Japan). (\url{https://en.brc.riken.jp}).
The related species of \emph{Mus caroli} CAROLI/EiJ and \emph{Mus
spretus} SPRET/EiJ were purchased from Jax labs
(\url{https://www.jax.org}) and \emph{Mus spicilegus} SPIC/Eij from
Riken.

All mice were housed UW-Madison Biotech and MSC facilities - following
the protocols. A breeding colony of wild derived \emph{Mus musculus}
mice sampled from Gough Island (GI)is maintained at UW Veterinary school
facilities. Mice were feed on dry standard breeder chow. Some strains
sunflower seeds, nestlets and larger cages were added to improve
fertility and litter survival. Adult mice were euthanized by CO
asphyxiation. Neonate and embryonic mice were euthanized by decapitation
following the protocols approved by the Institutional Animal Care and
Use Committee at the University of Wisconsin-Madison.

Over the course of data collection some breeding colonies mice were
moved from facilities. Additionally the GI strain was kept at a separate
facility. (evidence for environmental (room) effect on (MLH1 / SC
lengths / a variety of meiotic traits was tested for. We observed no
effect, Supplemental figure))

\section{Tissue Collection and
Immunohistochemistry}\label{tissue-collection-and-immunohistochemistry}

Spermatocyte spreads were collected and perpared as described in
(Peterson, Miller, and Payseur 2019) . The majority of mice used were
between 5 and 12 weeks, (supplemental table). However some strains had
problems breeding, so testes from older mice were collected. We found a
(small/no) age effect on MLH1 counts for male mice (Supplemental
Figure)).

The only difference between preparation of spermatocyte and oocyte
spreads was the volumns of the hypotonic buffer was 300 u l instead of
3ml.

The vast majority of oocyte data was collected from neonate litters/mice
between 5 to 48 hours. This approach maintain breeding pairing and still
result in prophase oocytes (cite timeline of oocytes in neonates).
(differences were assessed between embryo samples and neonates?).
Precise staging of embryos by checking copulary plugs was difficult/not
feasible in many of these wild derived strains due to their behavior.
Embryonic samples were collected when pregnancy noted in females,
embryos were staged based on (X table markers).

Meiocyte spreads were made following (Peters et al. 1997). For testes
the tunica was removed and whole testes was incubated in 3ml of
hypotonic solution for 45min. For ovaries after dissection, the pair of
ovaries were decapsulated in cold PBS and both were incubated in 300ul
of hypotonic solution for 45 min.

After incubation, gonads were transferred to sucrose solution -for
mastication to make a cell slurry which was transferred to 80ul of 2\%
PFA solution. Cells were fixed in this solution and dried in a humid
chamber at room temperature overnight. The next morning, slides were
treated with a photoflow (Kodak, cite) wash.

Staining / Immunohistochemistry

The same staining protocol was applied to spermatocyte and oocyte
spreads. The staining protocol was based on X with some adaptations, and
same as previously described in Peterson et al 2019.

\section{Image Processing}\label{image-processing}

Images of cells were capture using a Zeiss Axioplan 2 microscope with
AxioLab camera and Axio Vision software (Zeiss, Cambridge, UK).

Preprocessing, including cropping, noise reduction, and histogram
adjustments, was performed using Photoshop (v13.0).

For MLH1 pachytene characterization, Cells with a full karyotype (19
acrocentric bivalents and XY for spermatocytes or 20 acrocentric
bivalents for oocytes), distinct foci, and intact bivalents were
included for quantification. Reprocessing in Photoshop (cite). Image
file names were anonymized before manual scoring of MLH1 or DMC1 foci.

Hand measures data set generated by using ImageJ/Fiji (v1.52)
(Schindelin et al. 2012).

total Sc was measured (Wang et al. 2019)

Bivalent level measure were done (Peterson, Miller, and Payseur 2019)

\section{Statisical Analysis}\label{statisical-analysis}

Statistical analysis was performed in R (v3.5.2)(Team 2015).

The distributions of CO counts per cell was assessed for normailty
(supplemental figure). Mouse mean MLH1 count was used to get around the
bad discreet nature of count data. (normality was confirmed with X
distribution, supplemental figure).

framework for analyzing the MLH1 patterns 1. write equation

\$X \sim mouse average MLH1 \textasciitilde{} sex + subspecies +
random(strain) + \xi \$

mixed model, observations of mouse means of mlh1, all data with both
male and female observations

subspecies, sex and subsp*sex are fixed effects strain as random effect
(sample of a genetic background from subspecies) with domesticus male
set as the default

strain and sex interaction effect was included to ask / test if the
effect of sex varies by strain(genetic background)

this model allows us to estimate and test the fixed effects (sex and
subsp)

the random effect of strain in this model can be interpreted as the
average standard deviation due to the random strain effect and sex
strain,

From the mixed model the estimates of heterochiasmy (sexual dimorphism)
are a combination of subspecies and strain specific sex effects.

We can also compare the size of the sex effects between (subspecies) and
within (strain) mouse groups. (ie is the male or female strain effect
larger? is there more variance due to random strain effect in male or
females?)

I may have to do more research/reading on testing the random effects.

-specific statistical tests; (means, non-parametric kurts wallis
t-test?) (correlation coefficient (function)) linear models lm() from X
package.

-specific R packages -non-parametric (tests and measures for dealing
with count data) -hypothesis tested (size, DSB number)

Comparison/analysis of variation in MLH1 counts within and between
groups was done within a general linear/ mix model frame work, using X
package (cite). With X formula Co count \textasciitilde{} blah + blah,
with random strain specific sex effect.

(we used this frame work as proxy for `polymorphism' and `divergence' in
MLH1 counts across house mouse and closely related species.)

\section*{References}\label{references}
\addcontentsline{toc}{section}{References}

\hypertarget{refs}{}
\hypertarget{ref-peters_1997}{}
Peters, Antoine HFM, Annemieke W. Plug, Martine J. van Vugt, and Peter
De Boer. 1997. ``SHORT COMMUNICATIONS A Drying-down Technique for the
Spreading of Mammalian Meiocytes from the Male and Female Germline.''
\emph{Chromosome Research} 5 (1): 66--68.

\hypertarget{ref-peterson_2019}{}
Peterson, April L, Nathan D Miller, and Bret A Payseur. 2019.
``Conservation of the Genome-Wide Recombination Rate in White-Footed
Mice.'' \emph{Heredity}. Nature Publishing Group, 1--16.

\hypertarget{ref-fiji}{}
Schindelin, Johannes, Ignacio Arganda-Carreras, Erwin Frise, Verena
Kaynig, Mark Longair, Tobias Pietzsch, Stephan Preibisch, Curtis Rueden,
Stephan Saalfeld, and Benjamin Schmid. 2012. ``Fiji: An Open-Source
Platform for Biological-Image Analysis.'' \emph{Nature Methods} 9 (7):
676.

\hypertarget{ref-Rstudio}{}
Team, RStudio. 2015. ``RStudio: Integrated Development Environment for
R.'' Boston, MA. \url{http://www.rstudio.com}.

\hypertarget{ref-wang_2019_sc}{}
Wang, RJ, BL Dumont, P Jing, and BA Payseur. 2019. ``A First Genetic
Portrait of Synaptonemal Complex Variation.'' \emph{PLoS Genetics} 15
(8): e1008337--e1008337.


\end{document}
